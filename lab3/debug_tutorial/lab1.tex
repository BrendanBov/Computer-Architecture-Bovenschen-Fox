\documentclass{article}
\usepackage{amsmath, amsfonts, amsthm, amssymb}  

\usepackage{secdot}
\usepackage{epsfig}
\usepackage[T1]{fontenc}
\usepackage{epstopdf}
\usepackage{url}
\usepackage{rotating}
\usepackage{graphicx}
\usepackage{caption}
\usepackage{subcaption}
\usepackage{multirow}
\usepackage{setspace}
\usepackage{array}
\usepackage{fancyhdr}
\usepackage{lastpage}
\usepackage[T1]{fontenc}

\usepackage{geometry}
\geometry{letterpaper, left=1in, right=1in, top=1in, bottom=1in}

\pagestyle{fancy}
\fancyhf{}
\rhead{\thepage/\pageref{LastPage}}
\lhead{OSU ECEN 4243 - Computer Architecture - Spring 2019}
\rfoot{\LaTeX}


% ----- Identifying Information -----------------------------------------------
\newcommand{\myassignment}{Lab 1: Instruction-Level ARM Simulator}
\newcommand{\myduedate}{Assigned: Monday 1/14; Due \textbf{Monday 2/4} (midnight)}
\newcommand{\myinstructor}{Instructor: James E. Stine, Jr.}
\newcommand{\mytas}{TAs: Rachana Erra}
% -----------------------------------------------------------------------------

\begin{document}
\begin{center}
  {\huge \myassignment} \\
  {\large \myduedate} \\
  \begin{flushright}
  \myinstructor \\
  \mytas \\
  \end{flushright}
\end{center}

\section{Introduction}

For this assignment, you will write a C program which is an
instruction-level simulator for a limited subset of the ARM
instruction set. This instruction-level simulator will model the
behavior of each instruction, and will allow the user to run ARM
programs and see their outputs.  As we mentioned in class, these types
of queuing-type simulators are important in determining good choices
in microarchitecture design and subsequently its architecture.

This lab's objective is really accomplishes two tasks.  First, it
introduces you to software and the process in running code.  It
introduces you to the basics of compiling in C and what simulators
mean.  Second, it will introduce you to the ARM ISA.
% james  
All computer architects, programmers, and digitial
designers know how processors work by reading the reference manual.
And, this lab, will certainly introduce you to this.

The simulator will process an input file that contains an ARM
program. Each line of the input file corresponds to a single ARM
instruction written as a hexadecimal string. For example,
\verb+E282100A+ is
the hexadecimal representation of \verb+add r1, r2, #10+. We will
provide several input files. But you should also create additional
input files in order to test your simulator more comprehensively.

The simulator will execute the input program one instruction at a
time. After each instruction, the simulator will modify the ARM
architectural state: values stored in registers and memory. The
simulator is partitioned into two main sections: the (1) shell and the
(2) simulation routine. Your job is to implement the simulation
routine.

The source code for the lab are provided 
through D2L In the \verb+src/+ directory, we provide two
files (\verb+shell.c+ and \verb+shell.h+) that already implement the shell. 
There is a third file (sim.c) where you will implement the simulator
routine.

\section{Compilation Environment}

We will use the C language for our programming language and although
many of you may not have had experience in this, it is similar in
terms of its usage as Java.  Also, this lab gives you the underlying
framework and more complicated items and you are left to implementing
architectural items.  A wonderful introduction to the C
programming language is available in the text by Y. Patt and S. Patel,
which is also available on reserve in the Edmon Low
Library~\cite{Patt:2003:ICS:861709}.  More terse and complete texts
are available in other texts including the one of the original texts
on the subject by the inventors of C, Brian Kernighan and
Dennis Ritchie~\cite{Kernighan:1978:CPL:7519}.   There is also a great
reference from your ECEN 3233 textbook in Appendix
C~\cite{Harris:2015:DDC:2815529}\footnote{Available on D2L}. 

The problem with programming languages, like C and Java, is sometimes
you do not know how to compile someone else's program.  For Verilog in
Lab 0 we solved this with the DO file.  With the DO file you can
basically compile and simulate any program from another user provided
a DO file exists (i.e., \verb+vsim -do file.do+).  Most programming
languages solve this similarly
with a \verb+Makefile+.  A Makefile is a file that tells your system how to
compile your files for a specific programming language as well as
provide any needed command-line arguments necessary to have the
program compile correctly.  In fact, you can actually use a
\verb+Makefile+ with any type of compiled or interpreted language,
even Verilog. For this lab, a
\verb+Makfile+ is provided in the repository 
and all you have to do to compile the
program is type \verb+make+ provided the files are located in your
subdirectory that you are working in.  To compile the initial
skeleton, go to the subdirectory where you files exist and type
\verb+make+.  

There are many C compilers available, but you should have an ARMv8 GNU
binutils compilers installed on the Windows boxes in Endeavor 360.
You can also download from the link on D2L Brightspace and install on your
laptop/desktops (which is highly encouraged).  Although there are many
ways to interact with the compiler, I would recommend using the
command line interface.  Its easier to use the Command Line or Power
Shell through Windows 10.  There are other methods, but this is
probably the most recommended.  

\section{ARMsim}

The goal of this lab is to get you to think about how to model the ARM
ISA.  I have tried to give you a good framework for the simulator
written in C.  The ARMsim simulator is broken into two main C files:
\verb+shell.c+ and \verb+sim.c+.  The first file handles the
simulation from the user's point of view (i.e., the interface).  The
second file is the main simulator that makes sure instructions are
modeled corrected.  Each instruction is decoded and then \verb+isa.h+
handles the emulation in a function. 

\subsection{The Shell}

The purpose of the shell is to provide the user with commands to
control the execution of the simulator. The shell accepts one or more
program files as command line arguments and loads them into the memory
image. In order to extract information from the simulator, a file
named \verb+dumpsim+ will be created to hold information requested
from the simulator. The shell supports the following commands:

\begin{itemize}
\item go: simulate the program until it indicates that the simulator
  should halt. (As we define below, this is when a SWI instruction is
  executed with a value of \verb+0x0A+.)
\item run <n>: simulate the execution of the machine for n instructions.
\item mdump <low> <high>: dump the contents of memory, from location low
to location high to the screen and
the dump file (dumpsim).
\item rdump: dump the current instruction count, the contents of 
  R0 – R14, R15 (PC), and the CPSR to
the screen and the dump file (dumpsim).
\item input \verb+reg_num+ \verb+reg_val+: set general purpose
  register \verb+reg_num+ to value \verb+reg_val+.
\item ?: print out a list of all shell commands.
\item quit: quit the shell.
\end{itemize}

\subsection{The Simulation Routine}

The simulation routine carries out the instruction-level simulation of
the input ARM program. During the execution of an instruction, the
simulator should take the current architectural state and modify it
according to the ISA description of the instruction in the ARM
Architecture Reference Manual that is provided on the course
website. The architectural state includes the the general purpose
registers, the CPSR, and the memory image. The state is contained in the
following global variables:
\begin{verbatim}
#define ARM_REGS 16

typedef struct CPU_State {
  uint32_t REGS[ARM_REGS]; /* register file. */
  uint32_t CPSR; /* current program status register */
} CPU_State;

CPU_State STATE_CURRENT, STATE_NEXT;
int RUN_BIT;
\end{verbatim}
Furthermore, the simulator models the simulated system's memory. You
need to use the following two functions, which we provide, to access
the simulated memory:
\begin{verbatim}
uint32_t mem_read_32(uint32_t address);
void     mem_write_32(uint32_t address, uint32_t value);
\end{verbatim}
Note that in the ARM architecture, memory is
byte-addressable. Furthermore, we will
implement a little-endian architecture. This means that machine words
($32$ bits) are stored with the least-significant byte at the lowest
address, and the most- significant byte at the highest address. To
implement loads and stores of $8$-bit values, you will need
to use these $32$-bit memory access primitives (hint: be sure to modify
only the appropriate part of a $32$-bit word!).

In particular, you should call \verb+mem_read_32+ and
\verb+mem_write_32+ with only
$32$-bit-aligned addresses (i.e., the bottom two bits of the address
should be zero).
The simulator skeleton that we provide includes a function
named \verb+process_instruction()+ in the file \verb+sim.c+. This
function is called
by the shell to simulate one machine instruction. 
%You have to write
%the code for the function \verb+process_instruction()+ to simulate the
%execution of
%instructions. 
You can also write additional functions to make the
simulation modular (Keep in mind that you will probably be using the
code that you write in later labs in order to validate your work). 
We suggest
spending time to make your code easy to read and understand, for your
own benefit.

\subsection{What do you need to do?}
\label{instructions.sec}

Your job is to implement the process instruction() function in
sim.c. The process instruction() function should be able to simulate
the instruction-level execution of the following ARM instructions:

\begin{center}
\begin{tabular}{|c|c|c|c|c|c|c|} 
\hline
ADC  & ADD  & AND  & ASR  & B    & BIC  \\ \hline
BL   & CMN  & CMP  & EOR  & LDR  & LDRB \\ \hline
LSL  & LSR  & MOV  & MVN  & ORR  & ROR  \\ \hline
SBC  & STR  & STRB & SUB  & TEQ  & TST  \\ \hline
SWI  &      &      &      &      &      \\ \hline
\end{tabular}
\end{center}
For implementing these instructions, your tasks will be the
following. First, implement each of these instructions as specified
within the ARM Architecture Reference Manual accurately and
completely. 
Each instruction should be compatible with conditional
execution as described by the ARM manual. However, you only need to
implement a subset of conditions: EQ, NE, GE, GT, LT, LE, and AL.

In addition, for the Data Processing Instructions (again defined in
the reference manual), you must implement the S suffix for the
instructions. The S suffix (ADDS vs. ADD) allows the instruction to
set the CPSR's condition flag bits upon the execution of the
instruction. Although the CPSR has more functionality than just the
condition flags, you will only need to implement this functionality of
the CPSR. You must also implement both the immediate and register
operations for each Data Processing instruction. Finally, you should
implement the barrel shifting and register rotating functionality
defined in the reference manual as well.

It is important to note that for the SWI instruction, you only need to
implement the
following behavior: if the bottom byte of the instruction's value is
0x0A (decimal 10) when SWI is executed, then the go command should
stop its simulation loop and return to the simulator shell's
prompt. If the bottom byte is any other value, the instruction should
have no effect. No registers are modified in either case, except that
R15 (PC) is incremented to the next instruction as usual. The process
instruction() function that you write should cause the main simulation
loop to terminate by setting the global variable RUN BIT to 0. Also of
note, you should not worry about implementing any mode changes or
register switches on a SWI. Thus, you must only worry about one set of
registers.  

\textbf{NOTE:} ARM assumes that the PC value is actually equal to 
PC+8 when the
instruction at PC is being executed. This is because of ARM's pipeline
which keeps three instructions in flight at once. However, we do not
ask you to keep the incremented PC value. This means that whenever you
use an operation that requires the PC value (B, BL), you must use PC+8
as the base offset.

The accuracy of your simulator is your main priority. Specifically,
make sure the architectural state is correctly updated after the
execution of each instruction. We will test your simulator with many
input programs (some provided with the handout, some not) in order to
ensure that each instruction is simulated correctly.

In order to test that your simulator is working correctly, you should
run the input programs we provide you with and also write one or more
programs using all of the required ARM instructions that are listed in
the table above, and execute them one instruction at a time 
(run~$1$). You can use the rdump or mdump
command to verify that the state of the
machine is updated correctly after the execution of each instruction.

While the table appears to have many instructions, there are actually
only a few unique instruction behaviors with a number of minor
variations. You should tackle the instructions in groups: Data
Processing, Memory Instructions, Branches, and so on. The ARM Architecture
Reference Manual contains the official definition for each instruction
in this table (except for SWI, for which we provide a restricted
definition above). Please implement \underline{only} the $32$-bit
behavior of the instructions.

Finally, note that your simulator does not have to handle instructions
that we do not include in the table above, or any other invalid
instructions. We will only test your simulator with valid code that
uses the instructions listed above.  However, as noted below, adding
more instructions will give you some bonus points.  

\subsection{Some Guidance}

In order to give you some guidance and help you get started, since for
many this will be something different, much of the simulator is given
to you.  Also, dealing with a larger
program might be difficult, so we tried to avoid frustration with
writing code from scratch.   Most of the main functionality within the
simulator works well and should be functional (i.e., \verb+sim.c+ and
\verb+shell.c+).  However, if you notice within \verb+sim.c+ I put
an example of the instruction call for you.  For example, here is the
call for
\verb+ADD+:
\begin{verbatim}
  if(!strcmp(d_opcode,"0100")) {
    printf("--- This is an ADD instruction. \n");
    ADD(Rd, Rn, Operand2, I, S, CC);
    return 0;
  }
\end{verbatim}
These calls will be processed from \verb+isa.h+ and you can replicate
the piece of code to simulate the other instructions.  
I also gave you one free instruction that I implemented (i.e.,
\verb+ADD+) within the code.  

The ARM instruction set allows the condition field to be used with
conditionally executing instructions.  Typically, all instructions are
conditionally executed according to the state of the CPSR condition
codes and the instruction's condition field. This field (bits 31:28)
determines the circumstances under which an instruction is to be
executed. If the state of the C, N, Z and V flags fulfils the
conditions encoded by the field, the instruction is executed,
otherwise it is ignored.  
To make things easiier, I am not asking you
to implement these sort of instructions (e.g., addeq), but you do need
to implement direct comparisons (i.e., cmp r3, r2, r1).  However, this
would be a good thing to implement (i.e., addeq) for extra credit.

%It is also advisable to use other resources to help you.  
%The ARMsim
%simulator from University of Victoria should also help you with some
%of the simulations in terms of understand what is going on.  
%Please use other resources to help with the lab, including the
%green card and your textbook~\cite{Patterson:2016:COD:3027670}.  The
%ARMsim simulator should be installed on the PCs in
%the lab and can easily be installed to your laptop/desktop via the
%link on D2L. 

\subsection{Lab Files}

On D2L Brightspace, you will find a source code
distribution with three subdirectories arm2hex/, src/ and inputs/. In
src/, we are
providing you with the simulator skeleton as described above. You can
compile the simulator with the provided Makefile. In inputs/, we have
written some input files for you.  However, you should write more
input files in
order to be confident that your simulator is correct and that you
simulate all the instructions you are required to implement.
The README file in the arm2hex as well as information later described
in this document
describe how to assemble an ARM program with this script.  

\subsection{Compiling Code}

As stated previous, several examples are given in the \verb+input\+
directory of the repository.  Both assembly and hexadecimal versions
are given.  To compile the program into hex, you want to type the
following on the ECE server provided you copied \verb+arm2hex+ into
the directory you are compiling everything:
\begin{verbatim}
./arm2hex arm-none-eabi-as file.s file.x
\end{verbatim}

GNU tools are important to the use of computers and have been
instrumental in great advances in microarchitecture.  As discussed in
class, we will also use GNU tools for our work in this class.  All of
the ARM tools will have a program prefix of \verb+arm-none-eabi-+.
That is, if you want to the GNU binutils program \verb+gcc+, you
would type \verb+arm-none-eabi-gcc+.  If you remember that
compilers can compile things locally and/or with static or dynamic
libraries,
therefore, having environmental variables set up 
is extremely important.  To do this, you want to set up your
\verb+PATH+ and \verb+LD_LIBRARY_PATH+ variables.  The desktops in
ENDV 360 should be set up for you at the command prompt, however, you
can easily learn how to do this for any operating system through an
Internet search.  
Many of these tools are also available for different Operating Systems 
at the following URL:
\url{https://developer.arm.com/open-source/gnu-toolchain/gnu-rm/downloads}.
However, you have to set up the \verb+PATH+ and \verb+LD_LIBRARY_PATH+
yourself.  But, we encourage you to use these tools on your laptop and
desktop to make things easier for you to complete this lab.

\subsection{Resources}

If you have not done so already, we recommend that you 
become familiar with the ARM ISA and how the GNU compiler functions.
The ARM instruction set architecture is defined
in the referenced manual that we have provided on the course
website. 
However, the best reference is the list of ARM instructions in
Appendix B in your ECEN 3233 textbook~\cite{Harris:2015:DDC:2815529}.
This Appendix is great, because it lists all the instructions in a
concise manner and also provides a pseudo-language to understand each
instruction's operation called Register Transfer Language (RTL).
We may post
additional resources (clarifications, etc.) as required on the course
website. Finally, please do not hesitate to ask the TAs for help if you
become stuck
or if something is unclear! Take advantage of piazza,
office hours, and lab sections.  

\section{Handin}

You should electronically hand in your code (all files in the src/
directory) into D2L Brightspace through its dropbox.
Please contact the TAs for more help.  Your code should be
readable and well-documented. In addition, please turn in additional
test cases that you used in a inputs/ subdirectory. If you feel the
need to describe any additional aspects of your design in detail,
please include these in a separate README. 
Make sure that you test your simulator
extensively with a suite of test cases so that you are confident that
you have implemented all instructions correctly. This is for your
benefit in later labs!  

If you get stuck understanding the code or figuring out what is wrong,
it might help to either use a debugger or print out some variables.  C
has an excellent debugger called \verb+gdb+ and although it is a
little hard to first get started with, it is extremely powerful.
There is a tutorial posted on D2L that might help.  The other option
is print out certain points of the code to the screen with a
\verb+printf()+ statement.  This is not a great technique for
debugging, but sometimes it can solve problems quickly.

It is highly
recommended you try to tackle things early and get most of the work
done during the first week in lab.  It is also important to split the
work among your team to help get larger things done.  
%You are welcome
%to use ways to communicate with your partner, such as GitHub.  If you
%need help with
%how to do something with \verb+git+, please let us know.  
As I will
say for this and remaining labs that much of the work in these labs is
involved.  Therefore, do not procrastinate and get to it.  This lab
also has many practical and real world implications and can help you
understand how
to use C and programming languages.  

In order to grade the lab, we will use some unknown 
assembly code that
you will not see.  To get the maximum grade for this lab, your
simulator must be able to simulate all the instructions asked in
Section~\ref{instructions.sec}.  Your job is to really try to make
sure your simulator covers all possible instructions that this unknown
code may utilize.  

\subsection{Extra Credit}

There are many opportunities for improving the simulator.  Adding
extra features or more instruction support.  These extra items can
earn you extra points that may help you later in the semester.
Remember, if you try any extra credit or add any extra functionality
that it should be documented in your lab report as well as upload to
GitLab.   As with any extra credit, this should only be attempted
provided you have extra time with this course as well as your other
courses.  

\bibliographystyle{ieeetr}
\bibliography{ref}
\end{document}
